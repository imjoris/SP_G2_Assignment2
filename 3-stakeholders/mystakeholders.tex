% !TEX root = ../report.tex

\section{Stakeholders}
\label{sec:stakeholders}

Docker has the following stakeholders:
\begin{itemize}
\item Software developers
\item Software maintain
\item Cloud providers
\item The open container intiative
\item Docker developers
\item Plugin developers
\end{itemize}

%\item [Product owner] is the owner of the system. This stakeholder funds the project.  This affects the quality attributes usability and availability.
\subsection*{Software Developers}
The developers are those that use Docker to create their software systems. It is used by developers to deploy software and create architectures consisting of multiple containers. Developers using Docker for their system, expect it to be reliable and want it to have a good usability. 
%TODO source

\textbf{Concerns}
\begin{description}[labelindent=25pt,style=multiline,leftmargin=4.0cm,font=\normalfont\itshape]

\item[Usability] Since the software developers are using Docker, they want it to have good usability. This means it is easy to learn how to use Docker and it is not hard to work with. In fact, this is one of the main benefits of Docker, since it makes it easier to containerize applications, which could be done before, but was very difficult to do in practice.


\item[Functional correctness] Software developers care about the functional correctness, since bugs in the Docker software makes the development of their software more difficult. 

\end{description}

\subsection*{Software Maintainers}
Software maintainers are responsible for deploying the software and keeping the software product running. Docker is often used for software deployment (especially to the cloud). %TODO source
The software maintainers expect Docker to be reliable and portable.

\textbf{Concerns}
\begin{description}[labelindent=25pt,style=multiline,leftmargin=4.0cm,font=\normalfont\itshape]

\item[Reliability] Software maintainers are responsible for keeping the software working while it is deployed. They will only use Docker if it is reliable. It has to be tolerant against failing containers.

\item[Portability] The software maintainers want to be able to run Docker and its containers on a variety of different environments.

\item[Performance efficiency] The software maintainers want Docker to have a good performance. They want the resources of their servers to be used as efficiently as possible and therefore want Docker to have no overhead, like Virtual Machines often do have.

\end{description}



\subsection*{Cloud providers}
There are numerous cloud providers offering services which are based on Docker\footnote{\url{https://www.docker.com/partners\#/service}}. These cloud providers offer Container-based cloud computing, sometimes referred to as CaaS (Containers-as-a-Service).


\textbf{Concerns}
\begin{description}[labelindent=25pt,style=multiline,leftmargin=4.0cm,font=\normalfont\itshape]

\item[Installability] The cloud providers want to integrate Docker into their cloud computing architecture.

\item[Co-Existence] Docker has to share the environment with the existing architecture of the cloud providers.

\item[Reliability] The customers using the cloud providers' services expect a good reliability. If cloud providers are to implement Docker in their architectures, they need Docker to have good reliability.

% security?
\end{description}



\subsection*{The Open Container Initiative}

The Open Container Initiative\footnote{\url{https://www.opencontainers.org/}} was formed with the purpose of creating an open industry standard for container formats and runtime in June 2015.

They are interested in creating a formal, open, industry specification around container formats and runtime. This specification should be independent of particular clients/orchestration stacks, commercial vendors or projects and should be portable across a wide variety of operating systems, hardware etc.
% ^ according to about page

Docker has donated its container format and runtime, knows as `runC' to this initiative and is one of the members of the initiative, together with a lot of other members (including competing technologies, such as `rkt' from CoreOS)\footnote{A list is available at \url{https://www.opencontainers.org/about/members}}.

\textbf{Concerns}
\begin{description}[labelindent=25pt,style=multiline,leftmargin=4.0cm,font=\normalfont\itshape]

\item[Portability] The main goal of the initiative is to standardize the container format and runtime used also by the Docker project. This allows users of Docker to use the Docker containers with other container runtimes.

\item[Interoperability] Docker has to be able to work with the containers from the initiative.

\end{description}

\subsection*{Docker developers}
The Docker developers are the developers contributing to the Docker code base.

\textbf{Concerns}
\begin{description}[labelindent=25pt,style=multiline,leftmargin=4.0cm,font=\normalfont\itshape]

\item[Maintainability] These developers contribute new features and bugfixes to the existing code base. Therefore, they want the project to have good modifiability, such that additions and improvements can be realized without to much effort.

%\item[Functional correctness]

\end{description}

\subsection*{Plugin developers}
Docker allows extending its capabilities with plugins\footnote{\url{https://blog.docker.com/2015/06/extending-docker-with-plugins/}}. These plugins are created by the plugin developers.

\textbf{Concerns}
\begin{description}[labelindent=25pt,style=multiline,leftmargin=4.0cm,font=\normalfont\itshape]

\item[Adaptability\\(Portability)] The developers of the plugins want to extend the functionality of Docker in an effective and efficient way. 

\end{description}

\begin{table}[H] \centering
	\caption{Matrix of stakeholders concern.}
	\label{table:stakeholder_concern}
	\begin{tabular}{@{} cl*{10}c @{}}
		&  & \multicolumn{8}{c}{
		\textbf{Concerns}} \\[2ex]
		& \textbf{Stakeholder} 
			& 
			& \rot{Adaptability} 
			& \rot{Co-existence} 
			& \rot{Implementability}
			& \rot{Maintainability} 
			& \rot{Performance}
			& \rot{Portability}
			& \rot{Reliability}
			& \rot{Usability}
			\\
		\midrule
								%ada	co-ex	impl	maint	perfor	port	relia	usa	
& Software developer 	& &		&		&		&		&		&		&		X&		X 		\\
& Software main 		& &		&		&		&		&		X&		X&		X&				\\				
& Cloud providers 		& &		&		X&		X&		&		&		&		X&				\\			
& The open container 
initiative 				& &		&		&		&		&		&		X&		&				\\	
& Docker developers 
						& &		&		&		&		X&		&		&		&				\\				
& Plugin developers 
						& &		X&		&		&		&		&		&		&				\\				
\midrule
& Total					& &		1&	   1&	    1& 		1& 		1&		2&		3&		1		\\
\midrule
	\end{tabular}
\end{table}
