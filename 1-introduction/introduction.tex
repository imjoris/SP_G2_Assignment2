%!TEX root = ../report.tex
\clearpage
\chapter{Introduction}
\label{ch:introduction}
This document presents architecture recovery of Docker\footnote{\url{https://www.docker.com/}} by identifying software patterns and performing an evaluation of the architecture based on the identified patterns. This document is part of the Software Pattern assignment at the University of Groningen.

Docker is an open-source project that automates the deployment of applications inside software containers, by providing an additional layer of abstraction and automation of operating-system-level virtualization on Linux \cite{dockerdef}.


This project utilizes the IDAPO\footnote{Identifying Architectural Patterns in Open Source Software} process to recover the architecture \cite{idapo}. The PBAR\footnote{Pattern-Based Architecture Reviews} approach is used to perform the evaluation \cite{pbar}.
% explain a bit more

% Make sure this part is there, in the final document
% ===================================================
% The IDAPO process consists of twelve steps. Step 1 corresponds to Sections 2 and 3, in which the type and domain of Apache is identified. Section 4 is related to Steps 2 to 5 where used technologies and candidate patterns are examined.

% Steps 6 to 12 is an iterative process, which results in the documentation of used patterns. The results are document in Section 5. During these steps, documentation, source code and the components and connectors have been studied. Furthermore, the identification and validation of patterns and variants has been realized.

% Along with IDAPO, the PBAR process determines the high-level process of pattern-based recovery and the evaluation of the architecture. The findings of this evaluation are documented in Section 6. An overview of the process mapping can be found in Appendix A2.

The rest of the document is explained as follows. Chapter \ref{ch:context} gives brief explanation with regard to Docker. Chapter \ref{ch:stakeholders} elaborates on the stakeholders involved in the Docker project and its corresponding key-drivers. Patterns discovered in the Docker project are documented in chapter \ref{ch:patterns}. Evaluation is presented in chapter \ref{ch:evaluation}. Chapter \ref{ch:recommendations} gives several recommendation for the Docker project. Lastly, a conclusion is drawn in chapter \ref{ch:conclusion}.