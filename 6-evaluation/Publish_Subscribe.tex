% !TEX root = ../report.tex
\subsection{Publish-Subscribe}
\begin{figure}[H]
\centering
\includegraphics[scale=0.7]{6-evaluation/images/pubsub_frm.png}
\caption{Force Resolution Map for Publish-Subscribe pattern}
\label{fig:pubsub-frm}
\end{figure}
The publish subscribe pattern allows the event occurring in the Docker Registry to be triggered to the user who has previously subscribed to them. When those events happen the user receives a webhook notification.
% Mechanism :
% Endpoint, Http

As mentioned in the documentation\cite{docknotif}, Docker's implementation of this pattern places messages in an in-memory queue. Attempts to deliver messages are repeated until they are accepted. This means that when an endpoint is not available for a longer time, the message queue gets larger and drops messages. This has a negative impact on the reliability.

The portability is enhanced by this pattern because whatever the platform used everyone can subscribe to the notification system.

Security is not affected by this pattern.

%It can also be mentioned that the Publish-Subscribe patterns increases Scalability and enhances modifiability.

