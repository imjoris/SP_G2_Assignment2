%!TEX root = ../report.tex

\clearpage
\chapter{Conclusion}
\label{ch:conclusion}
% \section{Pattern-Based Architecture Reviews}
% \section{IDAPO}

% %sounds like we're about to die...
% \section{Final Words}

This document have presented the architecture recovery of Docker by identifying
software patterns and performing an evaluation of the architecture based on the
identified patterns according to their impact on the Docker's quality
attributes. This is done by implementing PBAR and IDAPO method. IDAPO method is
used to identify the patterns within the Docker architecture, while PBAR is
utilized to perform the architecture review.

We firstly start by grasping some basic information about Docker. The findings
were presented in the System Context chapter, which consists of brief
explanation about Docker and its ecosystem, along with the communities
supporting and developing it. We then figured out the stakeholders behind Docker
project and their concerns. Based on our findings, the stakeholders can be
grouped in to six groups, those are Docker developers, Software developers that
utilize Docker in their project, the Open Container Initiative, cloud providers,
Docker plugin developers, and software maintainers. The summary of stakeholders
and their concerns can be seen in Figure \ref{fig:stakeholders-quality}. Based
on their concerns, the key-drivers can be extracted and those are
\textit{portability, security, and reliability}.

Even if software designers or developers may not be aware of certain software
patterns, the patterns may still be present \cite{idapo}. The IDAPO process is
used to identify existing patterns in a open source software. Patterns recovery
is very useful for quantifying the software project on a high-level approach. By
knowing the patterns inside a particular software project it is also easier to
measure or evaluate it by their impact to the software's key-drivers. The IDAPO
consists of 12 iterative steps that aim to achieve a robust pattern recovery
inside the OSS project. The mapping of IDAPO process and the chapters of this
document is described in chapter \ref{ch:introduction}.

A typical evaluation of an architecture requires a lot of effort and cost, for
example, ATAM method. PBAR provides a lightweight architecture review process
that can be used where traditional architecture review methods would not because
of their high cost. PBAR focuses the evaluation on issues of the system's' key-
drivers. It makes use of discovered patterns within the architecture to properly
document issues. PBAR consists of four main steps, which the third step, the
review meeting, is actually a repetitive steps.

One of the reason why Docker is selected to be the target of this Pattern-based
recovery and evaluation is of its popularity. The purpose this software is to
make the deployment of distributed application easier. As stated in the
documentation, Docker provides an integrated technology stacks, which enables
the development and IT operation teams to build, ship, and run distributed
applications anywhere without having to know the platform running below it.

Our investigation showed that there are seven layers exists in Docker, although
more patterns are likely to be there as well. The detailed documentations of
those pattern are written in chapter \ref{ch:patterns}. In this project, we also
implemented some insight that we achieved when working with the first project of
Software Pattern. Hopefully, the enhanced insights and experiences of this
second assignment would be beneficial for us in the future.
