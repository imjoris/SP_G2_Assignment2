% !TEX root = ../report.tex

\section{Process View}
\label{sec:viewprocess}
This section discusses the system processes, how they communicate and the runtime behaviour of the system.

\subsection{Plugins}
\label{sec:processplugins}
Docker supports extending its capabilities using third-party plugins. At the moment of writing, Docker allows extending the functionality of the volumes and network subsystems, which are responsible for storing files outside containers and for inter-container communication over the network respectively. This will be extended in the future\cite{dockerplugindocs}.

A plugin runs in its own separate process and communicates with the Docker daemon using the REST API. In order for the Docker daemon to know about the plugins existence, a file has to be placed in a pre-defined directory. %todo explain exactly (sock file, etc file, json file ...)

In Figure~\ref{fig:activity_plugin} the process of using a plugin can be seen.
\begin{figure}[H]
\caption{An activity diagram showing the process of selecting a plugin (simplified)}
\centering
\includegraphics[scale=0.5]{4-softwarearch/images/plugins_activity.png}
\label{fig:activity_plugin}
\end{figure}

A user indicates that he/she wants to use a plugin by passing a command-line parameter to the Docker client when starting a container. If this parameter is present, then the daemon will start looking for the plugin in the plugin directory. If the plugin is not found, it returns an error to the user. 
When the plugin is found, the daemon will send a handshake to the plugin using a UNIX socket and the plugin will reply with a list of subsystems it implements. 
The Docker daemon will then use a Proxy to forward calls to this subsystem to the plugin process.
