% !TEX root = ../report.tex

\clearpage
\chapter{Pattern Documentation}
\label{ch:patterns}
This chapter describes the patterns used in Docker identified by us. We documented the patterns as described by Harrison et al.\cite{usingpatternscapture}, with added information about where the pattern was found (traceability).

\section{Client-Server}

% see https://docs.docker.com/engine/introduction/understanding-docker/

\begin{description}
\item [Traceability]~\\
The Client-Server pattern can be deducted from the online documentation\cite{dockerarchi}.

The Client-Server pattern can be deducted from the `What is Docker’s architecture?' section of the online documentation\cite{dockerarchi}.

\item [Source]~\\
Architectural patterns revisited -- a pattern language, P. 29 \cite{avgeriou2005architectural}

\item [Issue]~\\
It should be possible for Docker containers to be controlled remotely and a single interface should be able to control containers on multiple hosts (e.g. in the cloud).

Additionally, certain operating systems lack the underlying technologies necessary for running containers. For those OSes, it should be possible to call remote daemons running on operating systems which are supported. % /WindowsBashing

\item [Solution]~\\
Docker uses a \pattern{Client-Server} architecture. The client, a binary supplying a command-line interface, act as the primary interface for the user. The user enters commands into this client, which are then send to a server: the Docker daemon. 


\item [Assumptions/Constraints]~
\begin{itemize}
\item The versions of the client binary and the server binary should match. Different versions can cause problems.
\item All the services offered by the daemon have to be made available to the client using a REST interface.
\end{itemize}
~\\[-1.7cm]
\item [Rationale] ~\\
The daemon is a background process, which supplies the requested services to the client. The daemon exposes a REST interface.

For Docker, the client can be configured to connect to other daemon processes than the one running on the local machine. It can be configured to connect to remote Docker daemons as well, allowing the user to issue commands to daemons running remotely.

%todo move above to architecture chapter 

By separating the client and server it is possible to use the same client to issue commands to different daemons, running on different hosts.
It is also possible to use the client on operating systems that do not support running containers.

\item [Implications]~\\
The use of the \pattern{Client-Server} pattern results in two different executable binaries: a daemon and a client. 

The use of the \pattern{Client-Server} pattern increases the interoperability, since the client can send commands to daemons running on remote machines and the local machine.

Additionally, the portability is increased, since the client can run on Operating Systems that cannot run containers themselves.


\item [Related Patterns]~\\


\end{description}

\clearpage
\section{Layers}
% see https://www.docker.com/sites/default/files/what-is-vm-diagram.png

% \begin{figure}[H]
% \centering
% \includegraphics[scale=0.4]{5-patterns/images/what-is-vm-diagram.png}
% \caption{A layered overview of the Docker. Source \cite{whatisdocker}}
% \label{fig:layers-pattern}
% \end{figure}

\begin{description}
\item [Traceability]~\\
Layers is mentioned in the Docker architecture in the online documentation,
specifically in the explanation of Docker images\cite{dockerarchi}. \\
Furthermore, the user guide of Docker also elaborates on layers in the Docker images\cite{dockerimage}.

\item [Source]~\\
Architectural patterns revisited -- a pattern language, P. 29
\cite{avgeriou2005architectural}

\item [Issue]~\\
Docker emphasizes on developing containerization with almost no overhead. Running
multiple instances of Docker containers with the same image may end up in high
overhead. A sharing and reuse mechanism must be implemented to prevent overhead
and to save more space/resources.

\item [Assumptions/Constraints]~\\
An application running in a Docker container must have a CLI\footnote{Command Line
Interface} (bash), as Docker images are based on a Linux distribution.
%Wouter: not so sure about that, I saw a presentation once of somebody using volume mounts to have a program with GUI run in a Docker container

\item [Solution]~\\
Each Docker image consists of a stack of layers, which is read-only. The first
layer in the stack is a base image from a Linux distribution. Docker also
implements copy-on-write strategy. In this strategy, system processes that need
the same data share the same instance of data rather than having their own. A
copy will be made when a specific system process makes a change to the data (copy-on-write).

\item [Rationale] ~\\
Layers are very good in terms of sharing and reusability. In this way, overhead
can be avoided by sharing resources. A new image that uses same stack of layers
may reuse the layers rather than making a separate copy. This may result in
smaller sizes of Docker images.

\item [Implications]~\\
A Docker container will use Docker image as the base of it. This will remain as
read-only layers. Then, a Docker container will create another thin writable
layer on top of the image. Multiple containers that run based on the same Docker
image will not make their own copy of the image to prevent duplication.

This will end up in more efficient memory allocation (better performance efficiency).

\item [Related Patterns]~
\begin{itemize}
	\item Client-server
	\item Shared repository
\end{itemize}
\end{description}

% \clearpage


\section{Shared/Active repository}
%\textit{Can we consider the docker registry a shared repository?}
% Schemas
%http://fr.slideshare.net/Docker/https-dldropboxusercontentcomu20637798docker-meetup-freiburg
% http://blog.octo.com/en/docker-registry-first-steps/   http://fr.slideshare.net/egorpushkin/docker-demo   
% Because of Pull/¨Push can we talk about Pattern publish suscribe ?   Asynhcronous Queuing ?

The Docker registry is considered as a Shared Repository. \\
Two alternatives exist:DockerHub and Docker Trusted Registry. \\
Docker has a feature which is an option choosen by the user and can be configured : the Notification Sytem \cite{docknotif} which makes the Shared Repository an \textbf{Active Repository}.


\begin{description}
\item[Traceability]~\\
The Shared Repository pattern can be deducted from the online documentation : \cite{dockregistry} ``The Registry is a stateless, highly scalable server side application that \textbf{stores} and lets you \textbf{distribute} Docker images. A registry is a storage and content delivery system.''
Everything dealing with the registry is in the "Registry folder" in the Docker repository.

% File store.go , registry.go

\item[Source]~\\

Architectural Pattern Revisited - A Pattern Language, P.13 \cite{avgeriou2005architectural}

\item[Issue]~\\
Docker provided a way for the user to conrol the storage and distribution of images. \\
The user wants to be alerted of new events happening in the registry through notifications. % Develop ?

\item[Assumptions/ Contrainst]~\\

\item[Solution]~\\ %how does it work ?
Users interact with a registry by using docker push and pull commands. % develop

\item[Rationale]~\\ % in which way this pattern helps Docker? What is the goal ? KD
 After the integration of the Shared Repository Pattern each has one central repository containing their images and they can it access using a loggin.
 Users can also access images from others users : The registry is a sharing unit. \\
 
 \item [Implications]~\\
Users can get their own images from the public registry and also the ones created by other users. % share
The use of this pattern enhances Reusability, Changeability, Maintainability, Integrability because the Registry is central.

\item [Related Patterns]~\\
Publish Subscribe 
Brokered Authentication

\textit{Users can share and distribute images in the Docker Registry.}
 \begin{figure}[H]
 \centering
 \includegraphics[scale=0.7]{5-patterns/images/SharedRepo.png}
 \caption{Docker Registry and Shared Repository Patterns}
 \label{fig:docker-registry}
 \end{figure}
 
\end{description}

\section{Publish Subscribe}
% Deisgn Pattern : Event driven messaging 
\begin{description}

\item[Traceability]~\\
The notification mechanism of the Docker Registry uses the Publish Subscribe Pattern.

\item[Source]~\\
Pattern Oriented Software Architecture- Volume 1
%http://soapatterns.org/designpatterns/eventdrivenmessaging

\item[Issue]~\\ The user wants to be alerted about certain events/changes occuring in his registry.
The notification system needs a mechanism in order to send those events to the user.

\item[Assumptions/Contrainst]~\\ This pattern is used at a lower level in the Docker Registry/Active Repository Pattern.

\item[Solution]~\\

\item[Rationale]~\\ 

\quote {"Notifications are sent to endpoints via HTTP requests. Each configured endpoint has isolated queues, retry configuration and http targets within each instance of a registry. When an action happens within the registry, it is converted into an event which is dropped into an inmemory queue. When the event reaches the end of the queue, an http request is made to the endpoint until the request succeeds. The events are sent serially to each endpoint but order is not guaranteed."} % it seems like it's the way it works

\item[Implications]~\\ After the integration of the Publish-Suscribe pattern %BLABLABLA
The Publish Suscribe Pattern enhances Integrability,Modifiability because publisher and suscribers aren't directly connected.% develop

\item [Related Patterns]~\\
Shared/Active Repository Pattern


\end{description}

\begin{figure}[H]
\centering
\includegraphics[scale=0.7]{5-patterns/images/RegistryPS.png}
\caption{Events managing- Publish Subscribe Pattern}
\label{fig:publish-subscribe}
\end{figure}

\section{Brokered Authentication}

\begin{description}

\item[Traceability]~\\
From the v2 of Docker the authentication is done through a central service. \\
The Brokered Authentication pattern can be deducted from the source code through the "auth.go" and the "token.go" files from the Docker Registry Github Repository. \\
Docker Authentication \cite{dockauth}
%https://docs.docker.com/registry/spec/auth/jwt/

The main functions for authentication are : \\
func tryV2TokenAuthLogin(authConfig *types.AuthConfig, params map[string]string, registryEndpoint *Endpoint) \\
func loginV2(authConfig *types.AuthConfig, registryEndpoint *Endpoint, scope string) 
% They allow the 

\quote {This service is used by the official Docker Registry to authenticate clients and verify their authorization to Docker image repositories.}

\item[Source]~\\
Brokered Authentication Pattern\cite{brokeredauth} \\
%https://msdn.microsoft.com/en-us/library/ff650014.aspx

\item[Issue]~\\ The Docker Registry needs an authentication system in order to ensure security when allowing pull and push commands from the user.

\item[Assumptions/Contrainst]~\\ 

    Registry clients which can understand and respond to token auth challenges returned by the resource server.
    An authorization server capable of managing access controls to their resources hosted by any given service (such as repositories in a Docker Registry).
    A Docker Registry capable of trusting the authorization server to sign tokens which clients can use for authorization and the ability to verify these tokens for single use or for use during a sufficiently short period of time.


\item[Solution]~\\ 
An authentication broker that both parties trust independently issues a security token to the client. There is no direct relationship between the user and the registry.
"An authentication broker with a centralized identity store assumes the responsibility for authenticating the consumer and issuing a token that the consumer can use to access the service."
By using the Brokered Authentication the access to the Docker Registry is controlled for each user.

%The consumer submits a request with credentials to the authentication broker (1), which the broker authenticates against a central identity store (2). The broker then responds with a token (3) that the consumer can use to access Services A, B, and C (4), none of which require their own identity store.  % sum up in my own words
% Explaining the steps
%In the Docker registry case the central identity store is the Authorization service from the picture.

\item[Rationale]~\\ 

\item[Implication]~\\ After the integration of the Brokered authentication the textbf{Security} for the registry is ensured. % why BLABLABLA but single point of failure
The authentication broker manages trust centrally. This eliminates the need for each client and service to independently manage their own trust relationships.

\item [Related Patterns]~\\
Shared/Active Repository Pattern

\end{description}

\begin{figure}[H]
\centering
\includegraphics[scale=0.7]{5-patterns/images/Authentication.png}
\caption{Authentication process of the Docker registry }
\label{fig:auth-process}
\end{figure}

\section{Plugin}
\label{sec:pattern-plugin}
%TODO image
\begin{description}

\item [Traceability]~\\
The existence of Docker Plugins becomes apparent from it's documentation at \cite{dockerplugindocs}.

Additionally, the directories \verb|docker/pkg/plugins/| \footnote{\url{https://github.com/docker/docker/tree/master/pkg/plugins}} and \verb|  docker/daemon/graphdriver/plugin.go| \footnote{\url{https://github.com/docker/docker/blob/master/daemon/graphdriver/plugin.go}} (among others) in the project's repository contain the code for discovering plugins and the interfaces the plugins should implement.

\item [Source]~\\
Patterns of Enterprise Application Architecture, P. 499 \cite{eaa}

\item [Issue]~\\
The users of Docker want to have customization, by extending Docker with third party custom-built tools. This customization means that third parties should be able to write plugins that extend Docker's core functionality.\cite{dockerpluginblog}

The implementation of such plugins is only available at runtime.

\item [Assumptions/Constraints]~
\begin{itemize}
\item Plugins can only extends the functionality of the components of Docker, that have an interface that plugins can implement.
\end{itemize}


\item [Solution]~\\
Docker uses the Plugin pattern to link the implementation of the interfaces of several extendable components with third-party implementation at runtime.

\item [Rationale] ~\\ % https://docs.docker.com/engine/extend/plugin_api/
Docker discovers plugins by looking for .sock, .spec or .json files in the plugin directories on the host system. These files describe how Docker can communicate with the plugins using the REST API (usually via a Unix socket).

The plugins themselves run as separate processes on the same host as the Docker daemon and implement an HTTP server listening for requests from the Docker daemon. After a user requires a plugin (this is indicated e.g. as a command-line parameter when starting a container using the Docker client) Docker uses the discovery algorithm (see also Section~\ref{sec:processplugins}). After that, Docker sends a handshake to the plugin and the plugin returns a list of which subsystems this plugin implements.


For these subsytems Docker will replace the default implementation by a Proxy, that forwards all calls over the REST interface to the plugin process.

\item [Implications]~\\
The use of the Plugins pattern means that the adaptability increases. 

\item [Related Patterns]~
\begin{itemize}
\item Proxy
\end{itemize}
\end{description}

\section{Proxy}
\begin{description}

\item [Traceability]~\\
The use of the \pattern{proxy} pattern becomes apparent from the source code in the repository on GitHub.
For example, the proxy for the Volumes plugin can be found in \verb|docker/daemon/graphdriver/proxy.go| \footnote{\url{https://github.com/docker/docker/blob/master/daemon/graphdriver/proxy.go}}.

\item [Source]~\\
Pattern-oriented Software Architecture - Volume 4, P.290 \cite{wiley4}

\item [Issue]~\\
The plugins are separate processes than the daemon and are only available at runtime. Therefore, it is impossible for the daemon process to access the services of the plugins directly.
% issue also for comms between client-server ??

\item [Assumptions/Constraints]~
\begin{itemize}
\item The plugins have to implement a server listening for requests from the daemon.
\end{itemize}

\item [Solution]~\\
Let the daemon only communicate with the plugins through a proxy. This proxy implements all the `housekeeping' functionality, like sending API requests and authentication. It has the same interface as the plugin. \\
Whenever a call is done from the daemon to the plugin, it goes via the proxy, which communicates this call using a REST API to the plugin process.

\item [Rationale] ~\\ 
Using the \pattern{proxy} pattern allows the daemon to communicate with the plugins, without requiring direct access to these plugins. \\
Also, because the subsystem component has the same interface as its proxy, the implementation of software using this component does not depend on whether the proxy is used or the original subsystem. 

\item [Implications]~\\
The use of the \pattern{proxy} pattern allows communication with the plugins, which increases the extendability. 
Because the communication with the process is not direct, there is some performance overhead.
%TODO: what about portability??

\item [Related Patterns]~
\begin{itemize}
\item Plugin
\end{itemize}

\end{description}

