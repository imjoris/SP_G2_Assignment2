% !TEX root = ../report.tex

\clearpage
%\setlist[description]{%
%	leftmargin=!,%
%	itemindent=!,%
%	labelsep=!,%
%	labelwidth=!,%
%	labelindent=!,%
%}
%
%
%restore default description style
\setlist[description]{style=nextline}
% see:
%http://ctan.cs.uu.nl/macros/latex/contrib/enumitem/enumitem.pdf
%{labelwidth=6cm,labelindent=30pt,style=multiline,leftmargin=5.5cm,font=\normalfont\itshape}

\chapter{Pattern Documentation}
\label{ch:patterns}
This chapter describes the patterns used in Docker. The patterns are documented as described by Harrison et al.\cite{usingpatternscapture}, including additional information about concerning where the pattern was found (traceability) in the code or documentation.

\section{Client-Server}
\begin{figure}[H]
\centering
\includegraphics[scale=0.55]{5-patterns/images/client-server.png}
\caption{Visualization of the client-server pattern used by the Docker clients (client) and the Docker daemon (server).}
\label{fig:clientserver-pattern}
\end{figure}
% see https://docs.docker.com/engine/introduction/understanding-docker/

\begin{description}
\item [Traceability] The Client-Server pattern can be deduced from the `What is Dockers architecture?' section of the online documentation\cite{dockerarchi}.\\
When listing the dependencies using the dependency tools, very few packages use the docker daemon. Looking at the high-level parts of docker, the main package that references the daemon is the docker container.
However, the daemon seems to be referenced quite a lot when looking at table \ref{table:deptablehighestin}(section \ref{sec:viewlogical}), that lists the packages with the highest amount of incoming references. \\
This shows that the container part of docker is heavily using the docker daemon.\\
It is remarkable that when listing the dependencies of the daemon. It seems to only reference the container. So clearly, the docker daemon is a main access point for modifying the containers.\\
The documentation says that only the docker api communicates with the daemon. Clearly there is no package using the docker daemon, but also the api does not seem to be referencing the docker daemon. This is because the docker api uses the daemon through means of HTTP requests. As is made clear when looking at the code in $$api/client/cli.go$$
This contains the line: 
$$client, err := client.NewClient(host, verStr, clientTransport, customHeaders)$$

Indicating that the client is referencing the daemon through means of sending HTTP requests to the daemon api.

%github.com/docker/docker/builder
%github.com/docker/docker/container
%github.com/docker/docker/daemon
%github.com/docker/docker/distribution
%github.com/docker/docker/docker
%github.com/docker/docker/dockerinit
%github.com/docker/docker/layer

\item [Source]~\\
Architectural patterns revisited -- a pattern language, P. 29 \cite{avgeriou2005architectural}

\item [Issue]~\\
Docker containers can not be be controlled remotely.
It should be possible for Docker containers to be controlled remotely and a single interface should be able to control containers on multiple hosts (e.g. in the cloud).

Additionally, certain operating systems lack the underlying technologies necessary for running containers. The inability to remotely control containers from these operating systems prevents these systems from using docker in any way.

\item [Solution]~\\
Docker uses a {Client-Server} architecture. The client, a binary supplying a command-line interface, acts as the primary interface for the user. The user enters commands into this client, which are then sent to a server: the Docker daemon.


\item [Assumptions/Constraints]~
\begin{itemize}
\item The docker daemon should support the API calls made by the client. This means that if the client is outdated, the daemon is assumed to have the legacy controllers needed to still provide the wanted functionality.

%\item The versions of the client binary and the server binary should match. Different versions can cause problems when changes to the API were made.
%either the docker and client versions match or the docker daemon contains the needed

\item The REST interface that the daemon offers is assumed to be available and accessible by the client
\end{itemize}
~\\[-1.7cm]

\item [Rationale] ~\\

By separating the client and server it is possible to use the same client to issue commands to different daemons, running on different hosts.
It is also possible to use the client on operating systems that do not support running containers.

\item [Implications]~\\
The use of the {Client-Server} pattern results in two different executable binaries: a daemon and a client. 

The use of the {Client-Server} pattern increases the interoperability, since the client can send commands to daemons running on remote machines and the local machine.

Additionally, the portability is increased, since the client can run on Operating Systems that are unable run containers themselves.

There is also an impact on security, because the communication between the client and the server needs to be secured. Because docker is composed of different layers, securing this communication can be done from a single place. This is discussed in section \ref{sec:layers-pattern}, which discusses the layers pattern.

\item [Related Patterns]~
\begin{itemize}
\item \ref{pattern:broker}
\end{itemize}

\end{description}

\section{Layers}
\label{sec:layers-pattern}
% see https://www.docker.com/sites/default/files/what-is-vm-diagram.png
Layers pattern is a architectural pattern that consists of several layers, which
each layer provides a set of services to the layer above and uses the services
of the layer below. Docker implements this pattern to separate the concerns of
each component with independent functionalities. Layers pattern implementation
of Docker, based on our investigation, is shown in Figure \ref{fig:layers-pattern}.
% \newglossaryentry{OS}{name=OS, description={Operating System}}

\begin{figure}[H]
\centering
\includegraphics[scale=0.5]{5-patterns/images/LayersPattern.png}
\caption{A layered overview of the Docker.}
\label{fig:layers-pattern}
\end{figure}

\begin{description}
\item [Traceability]~\\
Layers is implicitly mentioned in the Docker architecture in the online
documentation, specifically in the \textit{What is Docker’s architecture?}
section \cite{dockerarchi}.

\item [Source]~\\
Architectural patterns revisited -- a pattern language, P. 29
\cite{avgeriou2005architectural}

\item [Issue]~\\
Docker provides several features (e.g. hosting the image, creating image, running container, and daemon supervision). Each of the features vary in terms of the functionality they give. Single implementation of features will certainly result in abundant usage of resources, because hosting the image means there must be huge amount of disk space available.
%TODO joris: Rephrase sentence about the hosting of the image. Rephrase, make clear why.
%
Overlapping may also occur since multiple Docker instances may use the same image or configuration file, which could be stored and organized elsewhere. Docker also has several dependencies, especially the Linux kernel components (\texttt{namespaces} and \texttt{cgroups}) for OS level virtualization.
Furthermore, Docker mostly needs remote supervision or control, since Docker usually runs in the cloud. Therefore, the features must be separated into several independent components and their dependencies and communication have to be clearly separated and maintained.

\item [Assumptions/Constraints]~\\
Docker is able to run only in Linux kernel since it makes use of
\texttt{namespaces} and \texttt{cgroups} technology, which is only available in
Linux kernel. However, Docker is also possible to be run on other OS
although using virtual machine. The connection between local and remote
components are carried out through secured TCP/IP connection.

\item [Solution]~\\
Docker is separated into several components that have their own
functionalities, as can be seen in Figure \ref{fig:layers-pattern}. The main components are:
\begin{itemize}
    \item Containers
    \item Docker daemon
    \item Local Docker client
    \item Libcontainer
    \item Linux kernel
\end{itemize}

Everything inside the dotted box is located on the same host. Inter-process communication is utilized to communicate between Docker, containers, libcontainer, and Linux kernel components, but Docker daemon, plugins, and Docker client communicate using Docker API
\footnote{\url{https://github.com/docker/docker/blob/master/api/common.go}}. Remote connections are managed through secured TCP/IP based connection.

\item [Rationale] ~\\
Layers are very good in terms of sharing and reusability. In this way, several
Docker daemon that uses the same image can just fetch it through the Docker
registry. Using Docker registry also reduce disk space for storing images and
increase easiness of sharing images. Layers pattern also supports connection to
other layer, such as connection to plugins. It is also manageable to supervise
remote Docker daemon through remote Docker client.

\item [Implications]~\\
This pattern gives positive implication to portability as clear separation of
containers, Docker daemon, Docker registry, and Docker client makes Docker
portable. Utilizing Docker registry also promotes sharing images, which make it
easier to deploy an application without knowing what platform running below it.

A host is open to public in the communication of Docker
daemon and Docker registry and remote client. This pattern provides the ability to secure all the communication from single place. This can increase the security, because focus on security is reduces toa single place, meaning that strengthening the security there makes everything more secure. \\
However, a security breach will effect the entire system and not only a single component. \\
This security layer can be easily replicated, having a positive impact on reliability. The security is defined in a single place, and so the level of security that is defined there can be expected from each component.

\item [Related Patterns]~
\begin{itemize}
	\item Client-server
	\item Shared repository
\end{itemize}
\end{description}

% \clearpage


\section{Shared/Active repository}
%\textit{Can we consider the docker registry a shared repository?}
% Schemas
%http://fr.slideshare.net/Docker/https-dldropboxusercontentcomu20637798docker-meetup-freiburg
% http://blog.octo.com/en/docker-registry-first-steps/   http://fr.slideshare.net/egorpushkin/docker-demo   
% Because of Pull/¨Push can we talk about Pattern publish suscribe ?   Asynhcronous Queuing ?
 \begin{figure}[H]
 \centering
 \includegraphics[scale=0.55]{5-patterns/images/SharedRepo.png}
 \caption{The shared repository with brokered authentication}
 \label{fig:docker-registry}
 \end{figure}
Using the Docker files and the docker commands, personalized images can be created that support a specified project. Docker allows these images to be stored in a docker registry, earlier discussed in subsection \ref{subsec:dockerregistry}. Anyone with access to this registry can pull this image and use it.\\
Users of the application a specific project provides, are able to pull the image. This allows running the application, thereby obtain the desired functionality. The registry also allows the image to be easily shared with other developers working on any related project.\\
This makes the docker registry a shared repository.
%TODO joris: add concluding sentence: 'So because the image can be easily shared, it increases porta/mainta/sec... enz.


%TODO joris: is this relevant? not specific to the pattern, should be discussed in logical view or earlier?
% Docker itself hosts a few registries that can be used called ``DockerHub'' \cite{} and ``Docker Trusted''. 
%Two alternatives exist:DockerHub and Docker Trusted Registry. \\
The Docker registry also contains functionality for sending notifications when certain events occur \cite{docknotif}. This makes the docker registry an active repository as well.

\begin{description}
\item[Traceability]~\\
The Shared Repository pattern can be deducted from the online documentation : \cite{dockregistry} ``The Registry is a stateless, highly scalable server side application that \textbf{stores} and lets you \textbf{distribute} Docker images. A registry is a storage and content delivery system.''
Everything dealing with the registry is in the "Registry folder" in the Docker repository.

% File store.go , registry.go

\item[Source]~\\
Patterns of Enterprise Application Architecture, P.322 \cite{eaa}\\
Architectural Patterns Revisited - A Pattern Language, P.13 \cite{avgeriou2005architectural}\\
Pattern-oriented Software Architecture - Volume 4, P.202 \cite{wiley4}

\item[Issue]~\\
Docker provided a way for the user to control the storage and distribution of images. \\
The user wants to be alerted of new events happening in the registry through notifications. % Develop ?

\item[Assumptions/ Constraint]~\\

\item[Solution]~\\ %how does it work ?
Users interact with a registry by using docker push and pull commands. % develop

\item[Rationale]~\\ % in which way this pattern helps Docker? What is the goal ? KD
 After the integration of the Shared Repository Pattern each has one central repository containing their images and they can it access using a loggin.
 Users can also access images from others users : The registry is a sharing unit. \\
 
 \item [Implications]~\\
Users can get their own images from the public registry and also the ones created by other users. % share
The use of this pattern enhances Reusability, Changeability, Maintainability, Integrability because the Registry is central.

\item [Related Patterns]~\\
Publish Subscribe 
Brokered Authentication

\textit{Users can share and distribute images in the Docker Registry.}
 
\end{description}

\section{Publish Subscribe}
% Deisgn Pattern : Event driven messaging 
\begin{description}

\item[Traceability]~\\
The notification mechanism of the Docker Registry uses the Publish Subscribe Pattern \cite{docknotif}.

\item[Source]~\\
Pattern Oriented Software Architecture- Volume 1
%http://soapatterns.org/designpatterns/eventdrivenmessaging

\item[Issue]~\\ The user wants to be alerted about certain events/changes that occur in a registry.
The notification system needs a mechanism in order to send those events to the user.

\item[Assumptions/Contrainst]~\\ 
This pattern is used at a lower level in the Docker Registry/Active Repository Pattern.



\item[Solution]~\\

\item[Rationale]~\\ 

%TODO fix, remove the big quote. Use POSA or something

\begin{quote}
"Notifications are sent to endpoints via HTTP requests. Each configured endpoint has isolated queues, retry configuration and http targets within each instance of a registry. When an action happens within the registry, it is converted into an event which is dropped into an inmemory queue. When the event reaches the end of the queue, an http request is made to the endpoint until the request succeeds. The events are sent serially to each endpoint but order is not guaranteed."% it seems like it's the way it works
\end{quote} 

\item[Implications]~\\ %After the integration of the Publish-Suscribe pattern BLABLABLA
The Publish Suscribe Pattern enhances Integrability,Modifiability because publisher and suscribers aren't directly connected.% develop

\item [Related Patterns]~\\
Shared/Active Repository Pattern


\end{description}

\begin{figure}[H]
\centering
\includegraphics[scale=0.7]{5-patterns/images/RegistryPS.png}
\caption{Events managing- Publish Subscribe Pattern}
\label{fig:publish-subscribe}
\end{figure}

\section{Brokered Authentication}

\begin{description}

\item[Traceability]~\\
From the v2 of Docker the authentication is done through a central service. \\
The Brokered Authentication pattern can be deducted from the source code through the "auth.go" and the "token.go" files from the Docker Registry Github Repository. \\
Docker Authentication \cite{dockauth}
%https://docs.docker.com/registry/spec/auth/jwt/

The main functions for authentication are : \\
\begin{itemize}
\item func tryV2TokenAuthLogin(authConfig *types.AuthConfig, params map[string]string, registryEndpoint *Endpoint)
\item func loginV2(authConfig *types.AuthConfig, registryEndpoint *Endpoint, scope string) 
\end{itemize}
% They allow the 

\begin{quote}
This service is used by the official Docker Registry to authenticate clients and verify their authorization to Docker image repositories.
\end{quote}

\item[Source]~\\
Brokered Authentication Pattern\cite{brokeredauth} \\
%https://msdn.microsoft.com/en-us/library/ff650014.aspx

\item[Issue]~\\ The Docker Registry needs an authentication system in order to ensure security of pulling and pushing images.

\item[Assumptions/Contrainst]~\\ 

  %  \quote {Registry clients which can understand and respond to token auth challenges returned by the resource server. \\
   % An authorization server capable of managing access controls to their resources hosted by any given service (such as repositories in a Docker Registry). \\
   % A Docker Registry capable of trusting the authorization server to sign tokens which clients can use for authorization and the ability to verify these tokens for single use or for use during a sufficiently short period of time.} \\

%"An authentication broker with a centralized identity store assumes the responsibility for authenticating the consumer and issuing a token that the consumer can use to access the service."
\item[Solution]~\\ 
An authentication broker that both parties trust independently, issues a security token to the client. The client can use this token to authenticate himself. This means that there is no direct relationship between the user and the registry.

By using the Brokered Authentication the access to the Docker Registry is controlled for each user.

%The consumer submits a request with credentials to the authentication broker (1), which the broker authenticates against a central identity store (2). The broker then responds with a token (3) that the consumer can use to access Services A, B, and C (4), none of which require their own identity store.  % sum up in my own words
% Explaining the steps
%In the Docker registry case the central identity store is the Authorization service from the picture.

\item[Rationale]~\\ 

\item[Implication]~\\ After the integration of the Brokered authentication the textbf{Security} for the registry is ensured. % why BLABLABLA but single point of failure
The authentication broker manages trust centrally. This eliminates the need for each client and service to independently manage their own trust relationships.

\item [Related Patterns]~\\
Shared/Active Repository Pattern

\end{description}

\begin{figure}[H]
\centering
\includegraphics[scale=0.7]{5-patterns/images/Authentication.png}
\caption{Authentication process of the Docker registry }
\label{fig:auth-process}
\end{figure}

\section{Plugin}
\label{sec:pattern-plugin}
%TODO image
\begin{description}

\item [Traceability]~\\
The existence of Docker Plugins becomes apparent from its documentation at \cite{dockerplugindocs}.

Additionally, the directories \verb|docker/pkg/plugins/| \footnote{\url{https://github.com/docker/docker/tree/master/pkg/plugins}} and \verb|  docker/daemon/graphdriver/plugin.go| \footnote{\url{https://github.com/docker/docker/blob/master/daemon/graphdriver/plugin.go}} (among others) in the project's repository contain the code for discovering plugins and the interfaces the plugins should implement.

\item [Source]~\\
Patterns of Enterprise Application Architecture, P. 499 \cite{eaa}

\item [Issue]~\\
%TODO joris: make neat sentence
Docker users desire functionality not natively provided by docker. They want the ability of extending Docker with third party custom-built tools that do provide this missing functionality. \\
These custumizations mean that third parties are able to write tools, extending Docker's core functionality\cite{dockerpluginblog}.
The additional functionality that is added, is only usable during run time.
%TODO joris: add consequences of this issue (regarding key d's?).


\item [Assumptions/Constraints]~
\begin{itemize}
\item Plugins can only extend the functionality of the components of Docker if they have an interface that plugins can implement.
\end{itemize}

\item [Solution]~\\
Docker uses the Plugin pattern to link the implementation of the interfaces of several extendable components with third-party implementation at runtime.

\item [Rationale] ~\\ % https://docs.docker.com/engine/extend/plugin_api/

\item [Implications]~\\
The use of the Plugins pattern means that the adaptability increases, because plugins allow the application to be adapted with new features.

For the security, it means that there is extra communication over the API which has to be secured. The security of the plugins themselves cannot be guaranteed by Docker.

\item [Related Patterns]~
\begin{itemize}
\item Broker
\end{itemize}
\end{description}

\section{Broker}
\manuallabel{pattern:broker}{Broker}
\begin{figure}[H]
\centering
\includegraphics[scale=0.43]{5-patterns/images/broker.png}
\caption{Broker for communication between Docker daemon and a plugin.}
\label{fig:broker-pattern}
\end{figure}

\begin{description}
\item [Traceability]~\\
The use of the Broker pattern can be found in the proxy package in the source code:\\
\verb| docker/pkg/proxy/tcp_proxy.go|\footnote{\url{https://github.com/docker/docker/blob/master/pkg/proxy/tcp_proxy.go}}.

\item [Source]~\\
Pattern-oriented Software Architecture - Volume 4, P.237 \cite{wiley4} \\
Architectural Pattern Revisited - A Pattern Language, P.34 \cite{avgeriou2005architectural}

\item [Issue]~\\
Docker consists of multiple processes. For example, the client and daemon are separate processes and also the plugins are separate processes.

These processes have to communicate with each other. However, dealing with the challenges of inter-process communication in the application code would greatly increase the complexity.
The application components should be shielded form the details of the inter-process communication.

\item [Assumptions/Constraints]~
\begin{itemize}
\item All the communication between the components in different processes have to go via the Broker.
\end{itemize}

\item [Solution]~\\
Use a Broker to separate the logic for communication between the processes from the application functionality. Use a component-based model (using Proxies) to allow local components to invoke methods on components in remote processes as if they were local.

\item [Rationale] ~\\ 
The Broker pattern allows for and handles all the communication between the different processes. 

This means that the logic for inter-process communication does not have to be implemented on a per-component basis.

\item [Implications]~\\
Using the Broker pattern is good for the portability, because it does not matter if a component is part of a different process (running locally or remotely).

This pattern is also good for the security, since the Broker is a single entrypoint for all the inter-process communication, which makes it easier to secure.

Additionally, the interoperability (compatibility) is increased, because it becomes possible for two components (in two different processes) to exchange information. 

The Broker does however introduce a performance overhead, which decreases the performance efficiency.

%TODO: reliability??
\textit{Reliability?}

%TODO: what about portability??

\item [Related Patterns]~
\begin{itemize}
\item Plugin
\item Proxy
\item Client-Server
\end{itemize}

\end{description}


\section{Proxy}
\manuallabel{pattern:proxy}{Proxy}
\begin{description}

\item [Traceability]~\\
The use of the {proxy} pattern becomes apparent from the source code in the repository on GitHub.
For example, the proxy for the Volumes plugin can be found in \verb|docker/daemon/graphdriver/proxy.go| \footnote{\url{https://github.com/docker/docker/blob/master/daemon/graphdriver/proxy.go}}.

\item [Source]~\\
Pattern-oriented Software Architecture - Volume 4, P.290 \cite{wiley4}

\item [Issue]~\\
The plugins are separate processes than the daemon and are only available at runtime. Therefore, it is impossible for the daemon process to access the services of the plugins directly.
% issue also for comms between client-server ??

\item [Assumptions/Constraints]~
\begin{itemize}
\item The plugins have to implement a server listening for requests from the daemon.
\end{itemize}

\item [Solution]~\\
Let the daemon only communicate with the plugins through a proxy. This proxy implements all the `housekeeping' functionality, like sending API requests and authentication. It has the same interface as the plugin. \\
Whenever a call is done from the daemon to the plugin, it goes via the proxy, which communicates this call using a REST API to the plugin process.

\item [Rationale] ~\\ 

\item [Implications]~\\
The use of the {proxy} pattern allows communication with the plugins, which increases the extendability. 
The Proxy has a positive impact on the portability because the location of the component accessed through the proxy is transparent for the caller.
Because the communication with the process is not direct, there is some performance overhead.
%TODO: what about security??

\item [Related Patterns]~
\begin{itemize}
\item Plugin
\item \ref{pattern:broker}
\end{itemize}

\end{description}

