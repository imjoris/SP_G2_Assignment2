\section{Broker}
\manuallabel{pattern:broker}{Broker}
\begin{figure}[H]
\centering
\includegraphics[scale=0.43]{5-patterns/images/broker.png}
\caption{Broker for communication between Docker daemon and a plugin.}
\label{fig:broker-pattern}
\end{figure}

\begin{patdescription}
\item [Traceability]
The use of the Broker pattern can be found in the proxy package in the source code:\\
\verb| docker/pkg/proxy/tcp_proxy.go|\footnote{\url{https://github.com/docker/docker/blob/master/pkg/proxy/tcp_proxy.go}}.

\item [Source]
Pattern-oriented Software Architecture - Volume 4, P.237 \cite{wiley4} \\
Architectural Pattern Revisited - A Pattern Language, P.34 \cite{avgeriou2005architectural}

\item [Issue]
Docker consists of multiple processes. For example, the client and daemon are separate processes and also the plugins are separate processes.

These processes have to communicate with each other. However, dealing with the challenges of inter-process communication in the application code would greatly increase the complexity.
The application components should be shielded form the details of the inter-process communication.

\item [Assumptions/Constraints]
\begin{mynesteditemlist}
\item All the communication between the components in different processes have to go via the Broker.
\end{mynesteditemlist}

\item [Solution]
Use a Broker to separate the logic for communication between the processes from the application functionality. Use a component-based model (using Proxies) to allow local components to invoke methods on components in remote processes as if they were local.

\item [Rationale]  
The Broker pattern allows for and handles all the communication between the different processes. 

This means that the logic for inter-process communication does not have to be implemented on a per-component basis.

\item [Implications]
Using the Broker pattern is good for the portability, because it does not matter if a component is part of a different process (running locally or remotely).

This pattern is also good for the security, since the Broker is a single entrypoint for all the inter-process communication, which makes it easier to secure.

Additionally, the interoperability (compatibility) is increased, because it becomes possible for two components (in two different processes) to exchange information. 

The Broker does however introduce a performance overhead, which decreases the performance efficiency.

%TODO: reliability??
\textit{Reliability?}

%TODO: what about portability??

\item [Related Patterns]
\begin{mynesteditemlist}
\item Plugin
\item Proxy
\item Client-Server
\end{mynesteditemlist}

\end{patdescription}
