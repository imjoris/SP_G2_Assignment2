% !TEX root = ../report.tex
\section{Brokered Authentication}

\begin{figure}[H]
\centering
\includegraphics[scale=0.7]{5-patterns/images/Authentication.png}
\caption{Authentication process of the Docker registry }
\label{fig:auth-process}
\end{figure}
\begin{patdescription}

\item[Traceability]
From v2 of Docker onwards the authentication is done through a central service, using brokered authentication\cite{dockauth}.

The Brokered Authentication pattern can be deducted from the source code through the \verb|auth.go| and the \verb|token.go| files from the Docker Registry Github Repository.

The main functions for authentication are:
\begin{itemize}
\item \codeu{func tryV2TokenAuthLogin(authConfig *types.AuthConfig, params map[string]string, \\registryEndpoint *Endpoint)}
\item \codeu{func loginV2(authConfig *types.AuthConfig, registryEndpoint *Endpoint, scope string) }
\end{itemize}
~\\[-2.0cm]
\item[Source]
Brokered Authentication Pattern\cite{brokeredauth}
%https://msdn.microsoft.com/en-us/library/ff650014.aspx

\item[Issue] The Docker Registry needs an authentication system in order to ensure security of pulling and pushing images.

\item[Assumptions/Constraints] 
The registry must be able to verify the tokens issued by the authorization service.

  %  \quote {Registry clients which can understand and respond to token auth challenges returned by the resource server. \\
   % An authorization server capable of managing access controls to their resources hosted by any given service (such as repositories in a Docker Registry). \\
   % A Docker Registry capable of trusting the authorization server to sign tokens which clients can use for authorization and the ability to verify these tokens for single use or for use during a sufficiently short period of time.} \\

\item[Solution] 
An authentication broker that both parties trust independently, issues a security token to the client. The client can use this token to authenticate himself. This means that there is no direct relationship between the user and the registry.

%The consumer submits a request with credentials to the authentication broker (1), which the broker authenticates against a central identity store (2). The broker then responds with a token (3) that the consumer can use to access Services A, B, and C (4), none of which require their own identity store.  % sum up in my own words
% Explaining the steps
%In the Docker registry case the central identity store is the Authorization service from the picture.

\item[Rationale] 
By using the Brokered Authentication the access to the Docker Registry is controlled for each user.

The authentication broker manages trust centrally. This eliminates the need for each client and service to independently manage their own trust relationships.

\item[Implications] After the integration of the Brokered authentication the security for the registry is ensured. % why BLABLABLA but single point of failure

Using Broker Authentication (instead of Direct Authentication) increases the portability, because the authentication service is decoupled from the registry. The authentication service can run on a completely different system. 

The reliability does decrease, because the dependence on the authentication service means that if this service becomes unavailable, so does the registry.
\item [Related Patterns]
Shared/Active Repository Pattern

\end{patdescription}

