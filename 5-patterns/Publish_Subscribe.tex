% !TEX root = ../report.tex
\section{Publish-Subscribe}
\begin{figure}[H]
\centering
\includegraphics[scale=0.6]{5-patterns/images/PubSub.png}
\caption{Docker Registry's event manager and subscribed endpoints}
\label{fig:publish-subscribe}
\end{figure}
\begin{patdescription}

\item[Traceability]
The notification mechanism of the Docker Registry uses the Publish Subscribe Pattern \cite{docknotif}.

\item[Source]
Pattern Oriented Software Architecture- Volume 4, P.234 \cite{wiley4} \\
Architectural Patterns Revisited -- a pattern language, P.32
\cite{avgeriou2005architectural}
%http://soapatterns.org/designpatterns/eventdrivenmessaging

\item[Issue] The user wants to be alerted about certain events/changes that occur in a registry.
The notification system needs a mechanism in order to send those events to the user.

\item[Assumptions/Constraints] 
This pattern is used at a lower level in the Docker Registry/Active Repository Pattern.

\item[Solution]
The Publish-Subscribe pattern allows publisher to send messages to endpoints which are subscribed to these messages.

As can be seen in Figure~\ref{fig:publish-subscribe}, a listener listens for events and forwards these to the event manager. The event manager forwards messages of this event to all the subscribed endpoints over HTTP.

\item[Rationale] 
Using the Publish-Subscribe pattern allows users to configure endpoints, which will receive notifications about the events/changes that occur in the registry.

\item[Implications] %After the integration of the Publish-Suscribe pattern BLABLABLA
The Publish-Subscribe Pattern enhances modifiability because publisher and subscribers are decoupled.

The portability is increased using this pattern, because the decoupling of the publishers and subscribers (combined with the messages send over HTTP) means that the subscribers can be in a completely different environment than the registry itself.

\item [Related Patterns]
Shared/Active Repository Pattern


\end{patdescription}

