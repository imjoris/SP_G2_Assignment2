\section{Plugin}
\label{sec:pattern-plugin}
%TODO image
\begin{patdescription}

\item [Traceability]
The existence of Docker Plugins becomes apparent from its documentation at \cite{dockerplugindocs}.

Additionally, the directories \verb|docker/pkg/plugins/| \footnote{\url{https://github.com/docker/docker/tree/master/pkg/plugins}} and \verb|  docker/daemon/graphdriver/plugin.go| \footnote{\url{https://github.com/docker/docker/blob/master/daemon/graphdriver/plugin.go}} (among others) in the project's repository contain the code for discovering plugins and the interfaces the plugins should implement.

\item [Source]
Patterns of Enterprise Application Architecture, P. 499 \cite{eaa}

\item [Issue]
%TODO joris: make neat sentence
Docker users desire functionality not natively provided by docker. They want the ability of extending Docker with third party custom-built tools that do provide this missing functionality. \\
These custumizations mean that third parties are able to write tools, extending Docker's core functionality\cite{dockerpluginblog}.
The additional functionality that is added, is only usable during run time.
%TODO joris: add consequences of this issue (regarding key d's?).


\item [Assumptions/Constraints]
\begin{mynesteditemlist}
\item Plugins can only extend the functionality of the components of Docker if they have an interface that plugins can implement.
\end{mynesteditemlist}

\item [Solution]
Docker uses the Plugin pattern to link the implementation of the interfaces of several extendable components with third-party implementation at runtime.

\item [Rationale]  % https://docs.docker.com/engine/extend/plugin_api/

\item [Implications]
The use of the Plugins pattern means that the adaptability increases, because plugins allow the application to be adapted with new features.

For the security, it means that there is extra communication over the API which has to be secured. The security of the plugins themselves cannot be guaranteed by Docker.

\item [Related Patterns]
\begin{mynesteditemlist}
\item Broker
\end{mynesteditemlist}
\end{patdescription}

