% !TEX root = ../report.tex

\clearpage
\chapter{Recommendations}
\label{ch:recommendations}
Table \ref{tab:qa-overview} shows the overview of the patterns and the implications on the key drivers. According to this table, the docker architecture is doing very well at increasing support for the main key driver: portability, while showing no particular effect for the other key drivers: security and reliability.\\
Docker is used a lot in the cloud to host certain services. Networking capabilities are this essential to Docker and directly effect the key drivers. It seams however, that certain aspects of Docker might negatively influence the an otherwise positive aspect of a pattern. This means that some patterns arent used in their full potential, thus giving the opportunity to increase key drivers.
Though Docker is doing very well at positively influencing their main key driver, there are some recommendations that might improve their key drivers. These recommendations are discussed in this chapter \\

\section{Networking}
Networking is done using a broker pattern, yet modifying the configuration of the networking seems to be very hard. This is because the networking is done by the docker daemon, only accessible using the client. So modification of the networking of docker has to be supported by the docker daemon and the docker client in order to work.\\
The functionality is there, however. So improving the networking functionality might be achieved by delegating the task of modifying the  configuration using already available services. The daemon is only required to provide the security by authenticating the request using the brokered authentication. The docker daemon could, for example, delegate docker client commands to ssh commands in the networking configuration of the container. \\

\section{Connectivity}
Docker containers are lacking capabilities to connect to each other. It is possible to let a docker container use an other docker container, but the functionality is very limited. For example, it does not seem possible to link the containers in a circular way.\\
Though the key drivers are not directly effected, interoperability is decreased. An application can still build and run everywhere, but only as stand-alone application.\\
This is something that the stakeholders would really like to have. In the architecture there is a layer pattern that clearly defines the functionality a layer should have. However, these layers do block the ability to communicate with an other docker container. Making the layers more accessible might improve the interoperability container capabilities. This will probably require modifications of the layers, redefining their functions. Doing so means modifications throughout the entire system, because all other patterns (broker, client-server, Plugin, publish-subscribe) need these layer definitions in order to function.

\section{security}
The security that is provided by the docker hub is not open source. Nautilus is the tool that is used to scan the images and this uses proprietary code.\\ 
Though it might secure the docker hub, docker registries that are not hosted by docker lack this security. This can negatively influence the security of docker greatly.\\
If docker were to use open-source security solutions, the overall security of docker would increase because then security would be easily available to the registries that are hosted on-premise.\\
Docker recently stated however that \q{But as of right now there are no plans to open source the scanning techniques}.

\section{Windows support}
Because the Docker daemon uses Linux-specific kernel features, you are unable to run Docker native in Windows. \\
Docker could try and create a layer that can communicate with windows as well. Doing so might be a lot of work though, but if they really want Docker application to be able to ``run everywhere'', they need to provide this functionality.
